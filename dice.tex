%%%%%%%%%%%%%%%%%%%%%%%%%%%%%%%%%%%%%%%%%%%%%%%
%                                             %
% Free beer for the one that reads this first %
%                                             %
%%%%%%%%%%%%%%%%%%%%%%%%%%%%%%%%%%%%%%%%%%%%%%%


%% start of file `template.tex'.
%% Copyright 2006-2013 Xavier Danaux (xdanaux@gmail.com).
%
% This work may be distributed and/or modified under the
% conditions of the LaTeX Project Public License version 1.3c,
% available at http://www.latex-project.org/lppl/.



\documentclass[11pt,a4paper,arial]{moderncv}        % possible options include font size ('10pt', '11pt' and '12pt'), paper size ('a4paper', 'letterpaper', 'a5paper', 'legalpaper', 'executivepaper' and 'landscape') and font family ('sans' and 'roman')

% moderncv themes
\moderncvstyle{casual}                             % style options are 'casual' (default), 'classic', 'oldstyle' and 'banking'
\moderncvcolor{blue}                               % color options 'blue' (default), 'orange', 'green', 'red', 'purple', 'grey' and 'black'
%\renewcommand{\familydefault}{\sfdefault}         % to set the default font; use '\sfdefault' for the default sans serif font, '\rmdefault' for the default roman one, or any tex font name
%\nopagenumbers{}                                  % uncomment to suppress automatic page numbering for CVs longer than one page

\usepackage{fontspec}
\setmainfont{Arial}

% character encoding
\usepackage[utf8]{inputenc}                       % if you are not using xelatex ou lualatex, replace by the encoding you are using
%\usepackage{CJKutf8}                              % if you need to use CJK to typeset your resume in Chinese, Japanese or Korean

% adjust the page margins
\usepackage[scale=0.75]{geometry}
%\setlength{\hintscolumnwidth}{3cm}                % if you want to change the width of the column with the dates
%\setlength{\makecvtitlenamewidth}{10cm}           % for the 'classic' style, if you want to force the width allocated to your name and avoid line breaks. be careful though, the length is normally calculated to avoid any overlap with your personal info; use this at your own typographical risks...

% personal data
%\name{Daan}{Wendelen}
%\title{Application for Frostbite: Rendering Engineer}                               % optional, remove / comment the line if not wanted
\address{Ambachtstraat 17A}{3530 Helchteren}{Belgium}% optional, remove / comment the line if not wanted; the "postcode city" and and "country" arguments can be omitted or provided empty
\phone[mobile]{+32~471~46~08~68}                   % optional, remove / comment the line if not wanted
%\phone[fixed]{+2~(345)~678~901}                    % optional, remove / comment the line if not wanted
%\phone[fax]{+3~(456)~789~012}                      % optional, remove / comment the line if not wanted
\email{daanwendelen@gmail.com}                               % optional, remove / comment the line if not wanted
%\homepage{www.johndoe.com}                         % optional, remove / comment the line if not wanted
%\extrainfo{additional information}                 % optional, remove / comment the line if not wanted
%\photo[64pt][0.4pt]{picture}                       % optional, remove / comment the line if not wanted; '64pt' is the height the picture must be resized to, 0.4pt is the thickness of the frame around it (put it to 0pt for no frame) and 'picture' is the name of the picture file
%\quote{Some quote}                                 % optional, remove / comment the line if not wanted

% to show numerical labels in the bibliography (default is to show no labels); only useful if you make citations in your resume
%\makeatletter
%\renewcommand*{\bibliographyitemlabel}{\@biblabel{\arabic{enumiv}}}
%\makeatother
%\renewcommand*{\bibliographyitemlabel}{[\arabic{enumiv}]}% CONSIDER REPLACING THE ABOVE BY THIS

% bibliography with mutiple entries
%\usepackage{multibib}
%\newcites{book,misc}{{Books},{Others}}
%----------------------------------------------------------------------------------
%            content
%----------------------------------------------------------------------------------
\begin{document}
%-----       letter       ---------------------------------------------------------
% recipient data
\recipient{Uprise}{Dragarbrunnsgatan 36C\\753 20 Uppsala}
%\recipient{DICE}{S\"odermalmsall\'en 36\\118 28 Stockholm}
\date{2017-05-01}
%\opening{Dear,}
%\closing{With kind regards,}
%\enclosure[Attached]{curriculum vit\ae{}}          % use an optional argument to use a string other than "Enclosure", or redefine \enclname

\makelettertitle

~\\

\textbf {Application for back-end software engineer}

~\\
~

It was 2004 when a TV show featured this amazing game. You could fly in planes and shoot them down with your machine gun. The game was
called ``Battlefield Vietnam''. I was impressed. The next week I bought the game, but our PC could not run it.
When the family got a new computer, I could finally experience the game. I have been a Battlefield enthusiast ever since.

~\\
~

A distributed system of low-cost hardware can outperform most mainframes for a fraction of the cost.
They can perform this miracle by communicating and working together. But the need to communicate creates interesting challenges.
These obstacles can be overcome with creative solutions and by involving our partners. I would love to tackle these issues
as a member of a diverse team. This way I can learn, teach and ultimately fulfill my full potential.

~\\
~

As a professional Java developer, I gained experience in software engineering, teamwork, communication and customer interactions.
I am proactive, a critical thinker and a good listener that defends his ideas with rational arguments.
My most valuable talent is that I quickly grasp new ideas and that I can apply them in other contexts. This allows me to come up with unique solutions.
I embrace continuous learning to keep my technical skills sharp.

~\\
~

I would like to elaborate on my motivation and competences in a personal interview,

~\\
~

\textbf {Daan Wendelen}

\makeletterclosing

\end{document}


%% end of file `template.tex'.
